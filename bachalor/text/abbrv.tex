
% Абревиатуры
\label{abbrv}
\chapter*{Перечень условных обозначений и сокращений}
\addcontentsline{toc}{chapter}{Перечень условных обозначений и сокращений}

%--- Acronyms -----------------------------------------------------------------%
% how to use acronyms:
% \ac = use acronym, first time write both, full name and acronym
% \acf = use full name (text + acronym)
% \acs = only use acronym
% \acl = only use long text
% \acp, acfp, acsp, aclp = use plural form for acronym (append 's')
% \acsu, aclu = write + mark as used
% \acfi = write full name in italics and acronym in normal style
% \acused = mark acronym as used
% \acfip = full, emphasized, plural, used
%--- Acronyms -----------------------------------------------------------------%

\begin{acronym}
        \acro{pqm}[PQM]{Passive Queue Management --- алгоритм пассивного управления очередью}
        \acro{aqm}[AQM]{Active Queue Management --- алгоритм активного управления очередью}
        \acro{red}[RED]{Random Early Detection --- алгоритм случайного раннего обнаружения}
        \acro{ared}[ARED]{Adaptive RED --- самонастраивающийся RED}
        \acro{rared}[RARED]{Refinde ARED --- улучшенный ARED}
        \acro{ns2}[NS-2]{Network Simulator 2}
\end{acronym}


\newpage
