
% Первая глава работы 
\chapter{Теоретическое введение}
\label{chap1}

\section{Методы моделирования}
\label{chap1:sec1}

%\subsection{Аналитическое моделирование}

\emph{Аналитическое моделирование} --- процесс формализации реального объекта
и нахождение его решения в аналитических функциях. Модель,
сформулированная на языке математики, физики, химии или другой науки с
помощью системы специализированных символов с точными правилами
сочетаемости называется аналитической моделью, чаще всего они
представляются в виде формул, неравенств, линейных и нелинейных
уравнений, в том числе дифференциальных и интегро-дифференциальных
уравнений и их комбинаций. Применяется для анализа характеристик
модели, полученной по упрощенным аналитическим зависимостям.

%\subsection{Натурное моделирование}

\emph{Натурным моделированием} называют проведение исследования на настоящем
предмете с последующей обработкой результатов опыта на основе теории
подобия. Натруное моделирование разделяется на научный эксперимент,
комплексные испытания и производственный эксперимент. Научный
эксперимент характеризуется обширным применением средств
автоматизации, использованием весьма всевозможных средств обработки
информации, возможностью вмешательства человека в процесс выполнения
эксперимента.

%\subsection{Имитационное моделирование}

\emph{Имитационное моделирование} --- это способ исследования, при котором
исследуемая система сменяется моделью, с достаточной точностью
описывающей настоящую систему, с которой проводятся опыты с целью
извлечения информации об этой системе. Такую модель можно «проиграть»
во времени, как для одного испытания, так и заданного их множества.


\section{Алгоритмы управления очередями}
\label{chap1:sec2}

FIXME: ((написать про то, зачем эти алгоритмы нужны и где применяются))

Алгоритмы управления и обработки очереди разделяются на два основных класса:

\begin{itemize}
  \item \acf{pqm}
  \item \acf{pqm}
\end{itemize}

\acfi{pqm} --- класс алгоритмов, применяемых для обработки очередей, при котором при достижении порогового значения, алгоритм отбрасывает пакеты в соответствии с некоторым правилом конкретной реализации алгоритма. Данные алгоритмы просты в исполнении, однако лишены возможности адаптироваться под разные нагрузки. 

Примеры алгоритмов пассивного управления очередью:

\begin{itemize}
  \item DropTail --- отбрасывает пакеты с конца очереди
  \item DropHead --- отбрасывает пакеты с начала очереди 
  \item RandomDrop --- отбрасывает случайные пакеты из очереди 
\end{itemize}

% subsection Пример (end)









%%



%%% Local Variables:
%%% mode: latex
%%% coding: utf-8-unix
%%% TeX-master: "../default"
%%% End:
