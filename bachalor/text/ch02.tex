% Вторая глава работы 
\chapter{Моделирование алгоритмов}
\label{chap2}

\section{Эталонная модель}
\label{chap2:sec1}

В рамках анализа алгоритмов была созданна эталонная модель, в рамках которой будут проводиться тестирование. Описание моделируемой сети:

\begin{itemize}
        \item Сеть состоит из 10 TCP-источников и TCP-приемников, двух маршрутизаторов R1 и R2 между приемниками и источниками.
        \item Между TCP-источниками и первым маршрутизатором установлена задежрка в 20 мс.
        \item Между TCP-приемниками и вторым маршрутизатором также установлена задежрка в 20 мс.
        \item Между маршрутизаторами установлена задежрка в 15 мс. и очередью
                типа DropTail / RED / ARED (именно здесь и будет происходить
                сравнение изучаемых алгоритмов).
        \item Максимальный размер TCP-окна 32; размер передаваемого пакета 1000 байт; время моделирования --- 20 секунд.
\end{itemize}

Данную модель реализовать как в Mininet, так и в NS-2. Сравнение в разных средствах моделирования представлено для понимания того, как модель себя будет вести в идеальной среде (в рамках имитационного моделирования) и в реальной среде (в рамках натурного моделирования). 

Реализация данной эталонной модели представлено в прил.~\ref{app1}

FIXME: добавить код и комментарии к нему

%%

%%% Local Variables:
%%% mode: latex
%%% coding: utf-8-unix
%%% TeX-master: "../default"
%%% End:
