% Вторая глава работы 
\chapter{Моделирование алгоритмов}
\label{chap2}

\section{Эталонная модель}
\label{chap2:sec1}

В рамках анализа алгоритмов была созданна эталонная модель, в рамках которой
будут проводиться тестирование. Описание моделируемой сети:

\begin{itemize}

        \item Сеть состоит из 10 TCP-источников и TCP-приемников, двух
                маршрутизаторов R1 и R2 между приемниками и источниками.

        \item Между TCP-источниками и первым маршрутизатором установлена
                задежрка в 20 мс.

        \item Между TCP-приемниками и вторым маршрутизатором также установлена
                задежрка в 20 мс.

        \item Между маршрутизаторами установлена задежрка в 15 мс. и очередью
                типа DropTail / RED / ARED (именно здесь и будет происходить
                сравнение изучаемых алгоритмов).

        \item Максимальный размер TCP-окна 32; размер передаваемого пакета 1000
                байт; время моделирования --- 20 секунд.

\end{itemize}

В связи с тем, что только два основных алгоритма активного управления очередью
(RED, ARED) реализованы в сетевом стеке ядра Linux --- они же и будут
рассматриваться для сравнения. Иные алгоритмы требуют внесения изменения в
модуль ядра.

Данную модель реализовать как в Mininet, так и в NS-2. Сравнение в разных
средствах моделирования представлено для понимания того, как модель себя будет
вести в идеальной среде (в рамках имитационного моделирования) и в реальной
среде (в рамках натурного моделирования). 

\section{Реализация в NS-2} % (fold)
\label{chap2:sec2}

% section Реализация в NS-2 (end)

\section{Реализация в Mininet} % (fold)
\label{chap2:sec3}

% section Реализация в Mininet (end)


%%

%%% Local Variables:
%%% mode: latex
%%% coding: utf-8-unix
%%% TeX-master: "../default"
%%% End:
