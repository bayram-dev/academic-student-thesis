\chapter*{Введение}
\addcontentsline{toc}{chapter}{Введение}

Данное исследование посвящено анализу и сравнению алгоритмов управления
очередями на основе RED, реализованных с использованием программных средств NS2
и MiniNet. Основная цель исследования - изучить принципы работы и эффективность
алгоритмов RED путем моделирования их поведения в различных сетевых условиях. В
рамках проекта была предпринята попытка смоделировать поведение сетей с
использованием алгоритмов RED и сравнить их производительность с целью
определения оптимальных настроек для обеспечения качественной и надежной
передачи данных. Результаты моделирования предназначены для определения
оптимальных параметров алгоритмов управления очередями, которые могут
способствовать общему повышению производительности сетевых систем.

\section*{Актуальность  темы}

Важность этого исследования заключается в изучении механизма активного
управления очередями (AQM) в RED, который помогает оптимизировать распределение
сетевых ресурсов и обеспечивает соответствие требованиям пользователей к
скорости и надежности передачи данных. 

\section*{Цель работы:}
Целью данной выпускной квалификационной работы является исследование и анализ
алгоритмов семейства RED (Random Early Detection) для активного управления
очередью. Основная задача заключается в изучении принципов работы и
эффективности данных алгоритмов в контексте управления потоками данных в
сетевых маршутизаторах, а также сравнительный анализ различных алгоритмов в
рамках натурной и имитационной модели.

Для достижения поставленных целей будут проведены исследования и анализ
различных алгоритмов из семейства RED. В ходе работы будет проанализировано
влияние параметров алгоритмов RED на производительность и стабильность сетевых
систем, а также будет проведено сравнение эффективности различных вариантов
алгоритма RED.

Для достижения целей работы будет использовано программное обеспечение и
инструменты моделирования NS2 и Mininet. Основной фокус будет сосредоточен на
сравнительном анализе результатов моделирования и выборе оптимальных параметров
алгоритма RED для достижения наилучшей производительности.

Основными задачами работы будут:

\begin{itemize}
        \item Изучение принципов работы алгоритмов семейства RED.
        \item Сравнительный анализ эффективности различных вариантов алгоритма RED.
        \item Проведение экспериментов с использованием различных инструментов моделирования для оценки эффективности алгоритмов RED.
\end{itemize}

В итоге данной работы ожидается получение информации об алгоритмах семейства
RED и их применении в сетевых системах, а также оценка эффективности этих
алгоритмов с использованием различных инструментов моделирования. Полученные
результаты могут быть использованы для оптимизации управления потоками данных в
сетевых системах и повышения их производительности и стабильности.

В результате этого исследования мы стремимся получить представление об
алгоритмах в рамках RED и их применении в сетевых маршрутизаторах. Мы также
оценим эффективность этих алгоритмов с помощью различных инструментов
моделирования. Результаты, полученные в результате этого исследования, могут
быть использованы для оптимизации управления потоками данных в сетевых системах
и повышения их производительности и стабильности.


\section*{Краткое содержание работы}

Данная работа состоит из введения, трех основных разделов, списка литературы и
приложений. В первом разделе рассматриваются инструменты сетевого моделирования
и принципы их работы, а также краткий обзор проверяемых алгоритмов. Во втором
разделе представлен обзор алгоритмов RED и их реализации в моделях в средствах
моделирования NS-2 и Mininet. В третьем разделе подробно представлены
результаты моделирования, построены необходимые графики и сделаны выводы
относительно эффективности предложенных алгоритма с использованием двух
инструментов моделирования, обобщены результаты работы и сделаны основные
выводы.

%\section*{Краткое содержание выпускной работы}




%%% Local Variables:
%%% mode: latex
%%% coding: utf-8-unix
%%% TeX-master: "../default"
%%% End:
