\begin{abstract}
	Алгоритм управления очередью, который применяется в маршрутизаторах, играет важную роль в обеспечении качества обслуживания (QoS). В этой работе представляется оценка производительности на основе моделирования и сравнения некоторых популярных методов управления очередями, такие как случайное раннее обнаружение (RED) и DropTail с точки зрения размера очереди, задержки в очереди, изменения задержки в очереди, скорости отбрасывания пакетов и использования полосы пропускания. Моделирование проходит в рамках имитационного и натурного моделирования для понимания эталонной модели и реальных показателей. Результаты моделирования показывают, что алгоритмы семейства RED показывают лучшие результаты в отношении Drop Tail с точки зрения задержки в очереди, изменения задержки в очереди. 
\end{abstract}

\makeabstract  
  
%%% Local Variables: 
%%% mode: latex
%%% coding: utf-8-unix
%%% End: 
