\begin{abstract}
	Алгоритм управления очередью, который применяется к маршрутизатору, играет важную роль в обеспечении качества обслуживания (QoS). В этой работе представляется оценку производительности на основе моделирования и сравнения некоторых популярных методов управления очередями, такие как случайное раннее обнаружение (RED) и DropTail (и другие алгоритмы) с точки зрения размера очереди, задержки в очереди, изменения задержки в очереди, скорости отбрасывания пакетов и использования полосы пропускания. Результаты моделирования показывают, что алгоритмы семейства RED показывают лучшие результаты в отношении Drop Tail с точки зрения задержки в очереди, изменения задержки в очереди и (более низкой) скорости отбрасывания пакетов. 
\end{abstract}

\makeabstract  
  
%%% Local Variables: 
%%% mode: latex
%%% coding: utf-8-unix
%%% End: 
